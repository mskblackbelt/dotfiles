%% Header file cloned from https://github.com/wickles/latex-base

%%%%%%%%%%%%%%%%%%
%% CONTENTS
%%%%%%%%%%%%%%%%%%
% To-do / issues
% Packages
% Commands
% Special Symbols
% Environments
% More commands: Resizable delimiters
% More commands: Derivatives
% Useful templates
% Notes
%%%%%%%%%%%%%%%%%%
%%%%%%%%%%%%%%%%%%



%%%%%%%%%%%%%%%%%%
%% TO-DO / ISSUES
%%%%%%%%%%%%%%%%%%

%% packages
% TO-DO: replace certain packages
	% fix COOL's broken derivatives
	% fix COOl conflict with Symbol fonts
  % fix CoOL support for upright Greek function names
	% commath -- basically rewrite it, especially by way of mathtools package's \DeclarePairedDelimiter command
		% use \DeclarePairedDelimiter command to define delimited operators like \abs, \norm, etc. Enables stuff like \abs{x} for normal size, \abs*{x} for auto size, \abs[\big]{x} for specific size
% TO-DO: custom commands using package rotating.sty
	% \newcommand{\sideways}[1]{\begin{sideways} #1 \end{sideways}} % turn things 90 degrees CCW
	% \newcommand{\turn}[2][]{\begin{turn}{#2} #1 \end{turn}} % \turn[ang]{stuff} turns things arbitrary +/- angle
% TO-DO: find package for easily drawing mapping / algebraic / commutative diagrams..??
% TODO: Remove mathtools, cancel





%%%%%%%%%%%%%%%%%%
%% PACKAGES
%%%%%%%%%%%%%%%%%%


%% debugging / diagnostics
\RequirePackage[l2tabu,orthodox]{nag} % nags user about obsolete and improper syntax

\usepackage{xparse} % provides high-level interface for producing document-level commands
	% via \[Declare/New/Renew/Provide/etc]DocumentCommand
	% allows for more than one optional argument in commands

\usepackage{iftex,ifluatex,ifxetex,ifdraft} % check if a document is being processed with pdfTeX, or XeTEX, or LuaTEX
\newif\ifxetexorluatex % a new conditional starts as false, true if using XeTeX OR LuaTeX
\ifnum 0\ifxetex 1\fi\ifluatex 1\fi>0
   \xetexorluatextrue
\fi

%% fonts and encoding 

% standard and structural packages

\usepackage{bm} % provides \bm command for robustly bolding math characters

% TODO: Investigate necessity and compatibility of the following lines
%%%%%%%%%%%%%%%%%%%%%% Additional math fonts %%%%%%%%%%%%%%%%%%%%%%%%%%%%%%%%%%%
\usepackage{amssymb} % for \mathbb (upper case only), \mathfrak fonts
%%%%%%%%%%%%%%%%%%%%%%%%%%%%%%%%%%%%%%%%%%%%%%%%%%%%%%%%%%%%%%%%%%%%%%%%%%%%%%%%

\usepackage{microtype} % improves kerning in certain cases. 
	% recommended to disable protrusion in table of contents!

%% Latex interface 

\usepackage{fvextra} % for verbatim and comment environments with \Verb
\usepackage{letltxmacro} % provides \LetLtxMacro command for correct renaming of commands
\usepackage{etoolbox} % provides many useful programming tools, 
	% e.g. \ifdefempty{cs}{true}{false}

%% media interface

\usepackage{graphicx} % support the \includegraphics command and options
\usepackage{subfig} % Support for subfigures and subcaptions
	\captionsetup[subfloat]{position=bottom}
\makeatletter%
\@ifclassloaded{tufte-handout}% xcolor is pre-loaded with the tufte-latex package
  {}%
  {\usepackage[dvipsnames,svgnames,table,hyperref]{xcolor}
	% provides access to large number of colors and related features
	% see end notes for lists of available colors
	}%
\makeatother%
\usepackage{svg} % provides \includesvg command for svg figures. 
\usepackage{pgfplots} % for plotting in tikzpicture environment
\usepackage{tikzscale} % allows \includegraphics{*.tikz} and scaling of TiKZ images

%% math interface

\usepackage{amsmath} % for nice math commands and environments
\usepackage{array} % improves array support, esp. in tabular env. 
	% see also xtab.sty


\usepackage{booktabs} % allows for improved spacing in tabular env. 
	% use \toprule, \*midrule, \bottomrule instead of \hline
	% see also ctable.sty
\usepackage{bropd} % provides \br command which simplifies nesting of bracketed terms within one another
	% e.g. \br{\br{x-a}^2+\br{y-b}^2} produces \left[ \left( x-a \right)^2 + \left( y-b \right)^2 \right]
	% also provides rudimentary derivatives, but overwritten by custom definitions

\usepackage{chemmacros,chemfig} % for writing chemical formulas with \ch, e.g. \ch{AgCl2-} or \ch{^{227}_{90}Th+}
  % Also provides state variables and equations, 
  % \[mM]olar, \Torr, \atm, \cal, \cmc, \MolMass units,
  % also loads siunitx and chemformula
	\DeclareSIUnit\ppm{ppm}
  \sisetup{% siunit package options
			per-mode = symbol,%
			inter-unit-product=\ensuremath{{}\!\cdot\!{}},%
			uncertainty-mode = separate,%
			separate-uncertainty-units = single,%
			retain-explicit-plus,%
			list-final-separator={, and }%
      }%

\usepackage{physics} % provides streamlined interface with many commands to simplify
	% writing standard physics notation (bra-kets, derivatives, etc.)
	% differentials and derivatives: \dd[n]{x}, \dv[n]{f}{x}, \dv{x}, \pdv{f}{x}{y}, \var{Q}, \fdv{F}{g}
	% bra-ket notation: \bra, \ket, \braket, \dyad, \matrixel 
	% \qty(x) for delimited quantities
	% \abs, \norm, \eval, \order ,\comm, \acomm
	% vectors: \vb, \va, vu, \vdot, \cross, \grad, \div, \curl, \laplacian
	% largely replaces and adds basic math functions with auto-delimiters: trig, linear algebra, etc.
		% no inverse hyperbolic trig? 
	% in particular adds \tr, \Tr, \rank, \erf, \Res, \pv / \PV, \Re, \Im
	% auto padding text: \qq{string}
	% matrix quantities: \mqty(a & b \\ c & d) or \mqty[x \\ y], Pauli \pmat{n}, diagonal matrices \mqty(\dmat{1,2&3\\4&5}), anti-diagonal \admat

%% misc packages

\usepackage{datetime} % allows easy formatting of dates, e.g. \formatdate{dd}{mm}{yyyy}
\usepackage[inline]{enumitem} % allows for custom labels on enumerated lists
	% e.g. \begin{enumerate}[label=\textbf{(\alph*)}]
	% label options are: \alph, \Alph, \arabic, \roman, and \Roman
	% inline option creates '*' versions of enumerate, itemize, description 
		% which can be inlined within the text of a paragraph. 
% \usepackage{outlines} % provides 'outline[style]' environment, allowing for subitems in lists
	% e.g. \begin{outline} \1 item \2 subitem \3[A)] subsubitem \1 item \end{outline}
	% or with other style: \begin{outline}[enumerate], etc

% \usepackage{rotating}
	% provides environments for rotating arbitrary objects, e.g. sideways, turn[ang], rotate[ang]
	% also provides macro \turnbox{ang}{stuff}
%\usepackage{ctable} % allows for footnotes under table and properly spaced caption above 
	% must be loaded after tikz
	% incorporates (..?)
\usepackage{tcolorbox}
	\tcbuselibrary{skins, breakable, xparse, minted}
	% provides fancier boxes than regular \makebox, \fbox, etc.
	% e.g. \doublebox, \ovalbox, \shadowbox
	% Can use `\tcbuselibrary{listings}` to use the listings library, 
		% doesn't require a language to be defined. 

%% document interface 

\usepackage{footnote} % 
\usepackage{hyperref} % adds hyperlinks and outline to PDF documents
	\hypersetup{%
		pdfencoding=auto,%
		psdextra,%
		pdfusetitle,%
		colorlinks=true,%
		linkcolor=BrickRed, %
		citecolor=Green, %
		filecolor=Mulberry, %
		urlcolor=NavyBlue, %
		menucolor=BrickRed, %
		runcolor=Mulberry, %
		linkbordercolor=BrickRed, %
		citebordercolor=Green, %
		filebordercolor=Mulberry, %
		urlbordercolor=NavyBlue, %
		menubordercolor=BrickRed, %
		runbordercolor=Mulberry %
		} %
	% options enable enhanced unicode and math support in PDF outlines [causes conflict with \C command?]
\usepackage{cleveref} % provides \cref command which inserts contextually correct word in front of ref.
	% e.g. \cref{eq:myeq} --> Equation 1.2, or so
\usepackage{bookmark} % improves package hyperref's bookmarking. 
	% properties such as style and color can be set. Generates bookmarks in first run. 

%% font packages -- load fontenc, then inputenc, then lmodern. 
% see http://tex.stackexchange.com/a/44699
\ifxetexorluatex % if lua- or xelatex http://tex.stackexchange.com/a/140164/1913
	\usepackage{fontspec}
	\usepackage[math-style=ISO]{unicode-math} 
	\ifdraft{}{
		\setmainfont{STIX Two Text}
		\setmathfont{STIX Two Math}
	}
	\renewcommand{\vb}[1]{\symbf{#1}} % fix vector command from physics package to work with unicode-math
\else 
	\usepackage[T1]{fontenc} % improves output font encoding
		% T1: provides expanded Latin character encoding, with accents
		% T2A: provides Cyrillic character encoding for use in cyrillic math commands
	\usepackage[utf8]{inputenc} % provides full utf8 input encoding
	%\usepackage{lmodern} % loads Latin Modern fonts, enhanced versions of the Computer Modern (CM) fonts.
		% They have enhanced scaling, metrics and glyph coverage. 
		% See http://www.tug.dk/FontCatalogue/lmodern/
		% Also solves rasterized/aliased fonts issue on incomplete LaTeX installations without CM-super fonts
	\usepackage{stix2}
	\usepackage{cmap} % provides better copy-pasteable output text. 
		% see for details: http://tex.stackexchange.com/a/64198
		% can also use mmap.sty, but not compatible with fonts other than Computer Modern
		% see for details: http://tex.stackexchange.com/questions/74630/
	\input{glyphtounicode}	% these two commands improve glyph output for non-T1 encodings
	\pdfgentounicode=1		% and this
\fi



%% load later packages
\usepackage{lineno} % provides line numbers in main text for reference and peer review
	% activated by calling \linenumbers in document
\usepackage[textsize=footnotesize]{todonotes}


%%%%%%%%%%%%%%%%%%
%% COMMANDS
%%%%%%%%%%%%%%%%%%


% functions and operators
% some functions deprecated by physics.sty
\newcommand{\code}[2][]{ % for typesetting monospaced code
	%\ifthenelse{\equal{#1}{}}{ % if description is empty
	%\ifdefempty{#1}{
	\ifstrempty{#1}{
		\texttt{#2}
	}{ %else
		\colorbox{#1}{\texttt{#2}}
	}
}
\newcommand{\mtext}[1]{{\textnormal{#1}}} % for writing text within math mode, e.g. for subscripts
\DeclareMathOperator{\sgn}{sgn}
\DeclareMathOperator{\erfc}{erfc}
\DeclareMathOperator{\GammaFunc}{\symup{\Gamma}}
\newcommand{\iu}{\TextOrMath{$\mtext{i}$}{\mtext{i}\mkern1mu}}
%
%
% %%%%%%%%%%%%%%%%%%%%%
% %% SPECIAL SYMBOLS %%
% %%%%%%%%%%%%%%%%%%%%%
%
% % special characters and shortcuts
\newcommand{\const}{{\textit{const}}} % shorthand for constant. works in math mode with proper kerning
\newcommand{\bigO}{\mathcal{O}} % big O notation. is there a \O?
\newcommand{\Lag}{\mathcal{L}} % fancy Lagrangian
\newcommand{\Ham}{\mathcal{H}} % fancy Hamiltonian
\newcommand{\AmpliM}{\mathcal{M}} % fancy M for scattering amplitudes
\newcommand{\AmpliA}{\mathcal{A}} % fancy A for scattering amplitudes
\newcommand{\emf}{\mathcal{E}} % fancy E for EMF
% fields and spaces
\newcommand{\fatemptyset}{\varnothing} % alternative to the thin character \emptyset
\newcommand{\reals}{\mathbb{R}} % real numbers
\newcommand{\complexes}{\mathbb{C}} % complex numbers
\newcommand{\ints}{\mathbb{Z}} % integers
\newcommand{\nats}{\mathbb{N}} % natural numbers
\newcommand{\irrats}{\mathbb{Q}} % irrationals
\newcommand{\quats}{\mathbb{H}} % quaternions (a la Hamilton)
\newcommand{\euclids}{\mathbb{E}} % Euclidean space
%
% %%%%%%%%%%%%%%%%%%
% %% USEFUL TEMPLATES
% %%%%%%%%%%%%%%%%%%
%
% \begin{comment}
%
% % make a professional looking table
% %\usepackage{booktabs}\\
% \begin{table}
% %\centering % optional
%   \begin{tabular}{llr}
%     \toprule
%     \multicolumn{2}{c}{Item} \\
%     \cmidrule(r){1-2}
%     Animal    & Description & Price (\$) \\
%     \midrule
%     Gnat      & per gram    & 13.65      \\
%           &    each     & 0.01       \\
%     Gnu       & stuffed     & 92.50      \\
%     Emu       & stuffed     & 33.33      \\
%     Armadillo & frozen      & 8.99       \\
%     \bottomrule
%   \end{tabular}
% \caption{My neat table}
% \end{table}
%
% % handle figures using graphicx package
% % example - place a figure
% \begin{figure}
% \centering
% \includegraphics[width=0.8\textwidth]{myfile.png}
% \caption{This is a caption}
% \label{fig:myfigure}
% \end{figure}
%
% % handle subfigures using subcaption package
% % example - place three subfigures with caption and subcaptions
% \begin{figure}
% 	\centering
% 	\begin{subfigure}[b]{0.3\textwidth}
% 		\includegraphics[width=\textwidth]{gull}
% 		\caption{A gull}
% 		\label{fig:gull}
% 	\end{subfigure}%
% 	~ %add desired spacing between images, e. g. ~, \quad, \qquad, \hfill etc.
% 		%(or a blank line to force the subfigure onto a new line)
% 	\begin{subfigure}[b]{0.3\textwidth}
% 		\includegraphics[width=\textwidth]{tiger}
% 		\caption{A tiger}
% 		\label{fig:tiger}
% 	\end{subfigure}
% 	~ %add desired spacing between images, e. g. ~, \quad, \qquad, \hfill etc.
% 	%(or a blank line to force the subfigure onto a new line)
% 	\begin{subfigure}[b]{0.3\textwidth}
% 		\includegraphics[width=\textwidth]{mouse}
% 		\caption{A mouse}
% 		\label{fig:mouse}
% 	\end{subfigure}
% 	\caption{Pictures of animals}\label{fig:animals}
% \end{figure}
%
%
% % draw plots with pgfplot package
% % example - draws plot of (???)
% \pgfplotsset{compat=1.3,compat/path replacement=1.5.1}
% \begin{tikzpicture}
% \begin{axis}[
% extra x ticks={-2,2},
% extra y ticks={-2,2},
% extra tick style={grid=major}]
% \addplot {x};
% \draw (axis cs:0,0) circle[radius=2];
% \end{axis}
% \end{tikzpicture}
%
%
% \end{comment}
% %%%%%%%%%%%%%%%%%%
%
%
% %%%%%%%%%%%%%%%%%%
% %% NOTES
% %%%%%%%%%%%%%%%%%%
%
%
%% Note on math environments
% 'equation' and 'align' are basic standards
%	 use '&' to for alignment in 'align'
%	 multi-line environments all end lines with '\\' (except for last line)
%	 generally all lines are tagged except in 'split' and '*' versions of environments
%	 add \label{txt} after line to give LaTeX-label, \tag{txt} to change textual label
% use 'gather' to display multiple center-aligned lines (no manual alignment)
%	 tags/labels every line
% use 'multline' to left-align first line, right-aline last line, and center-align all lines in between (no manual alignment)
%	 only tags/labels last line
% use 'split' within 'equation' or 'align' to give single middle-aligned tag/label to the contained lines
%	 use '&' for alignment within 'split'
%	 no '*' version
%
%% Note on spacing
% https://tex.stackexchange.com/questions/74353
% 5) \qquad
% 4) \quad
% 3) \thickspace = \;
% 2) \medspace = \:
% 1) \thinspace = \,
% -1) \negthinspace = \!
% -2) \negmedspace
% -3) \negthickspace

%% Note on matrices
% https://en.wikibooks.org/wiki/LaTeX/Mathematics#Matrices_and_arrays
% use \matrix for simple n*n matrix. prefix with p, b, B, v, V for parentheses, square brackets, curly braces, vertical lines, double vertical lines respectively.
% use \smallmatrix for inline text. pre- and postfix with delimiters if desired.

%% Note on useful aliases
% \rightarrow = \to
% \leftarrow = \gets
% \ni = \owns = backwards \in
% \cup = union, \cap = intersect
% \sqcup / \sqcap for disjoint varieties
% \gg = >>, \ll = <<
% \vert = | (pipe, e.g. absolute value delimiter),
% \Vert = || (double pipes, e.g. norm delimiter)
% \sim = ~ (tilde)
% \overline = bar over the top. \bar achieves same effect but is fixed-width.
% \emptyset is thin (a zero with a slash). use \varnothing for fat version (defined as \fatemptyset above).
% AMS dots: \dotsc for comma dots, \dotsm for multiplicative dots,
	% \dotsb for binary operator dots, \dotsi for multiple integration dots,
	% \dotso for other dots

%% Note on primitives and wrappers
% TeX functional primitives
	% \let \foo = \bar assigns value of \bar to \foo at point of definition
	% \def \foo{\bar} assigns value of \bar at point of use
	% egreg recommends always using \LetLtxMacro{old}{new} instead
		% http://tex.stackexchange.com/a/29837/70383
% LaTeX functional primitives
	% \newcommand defines a new command, and makes an error if it is already defined.
	% \renewcommand redefines a predefined command, and makes an error if it is not yet defined.
	% \providecommand defines a new command if it isn't already defined.
% \DeclareMathOperator is a wrapper for \operatorname.
	% both have '*' versions for operators with limits

%% Note on fonts
% Use the following string to test readability of editor font: Illegal1 = O0
% Default LaTeX document classes only support font sizes of 10, 11, and 12 points.
	% Third party `extsizes' classes exist that allow 8-12, 14, 17, and 20 point fonts.
	% Available classes: extarticle, extreport, extbook, extletter, and extproc
% For text within math mode, always use \textnormal -- NOT \mathrm, \text, \textrm, etc.
	% this maintains font choices and ensures normal, upright text
% Text mode font sizes (small to large)
	% \tiny
	% \ssmall % (moresize.sty)
	% \scriptsize
	% \footnotesize
	% \small
	% \normalsize
	% \large
	% \Large
	% \LARGE
	% \huge
	% \Huge
	% \HUGE % (moresize.sty)
% Math mode font sizes for delimiters (smaller to larger)
	% \big
	% \Big
	% \bigg
	% \Bigg

%% pre-defined colors
% standard: black, blue, brown, cyan, darkgray, gray, green, lightgray, lime, magenta, olive, orange, pink, purple, red, teal, violet, white, yellow
%
% dvips: Apricot, Aquamarine, Bittersweet, Black, Blue, BlueGreen, BlueViolet, BrickRed, Brown, BurntOrange, CadetBlue, CarnationPink, Cerulean, CornflowerBlue, Cyan, Dandelion, DarkOrchid, Emerald, ForestGreen, Fuchsia, Goldenrod, Gray, Green, GreenYellow, JungleGreen, Lavender, LimeGreen, Magenta, Mahogany, Maroon, Melon, MidnightBlue, Mulberry, NavyBlue, OliveGreen, Orange, OrangeRed, Orchid, Peach, Periwinkle, PineGreen, Plum, ProcessBlue, Purple, RawSienna, Red, RedOrange, RedViolet, Rhodamine, RoyalBlue, RoyalPurple, RubineRed, Salmon, SeaGreen, Sepia, SkyBlue, SpringGreen, Tan, TealBlue, Thistle, Turquoise, Violet, VioletRed, White, WildStrawberry, Yellow, YellowGreen, YellowOrange